\section{Combination of experimental inputs}
\label{Sec:Combination}

\sloppy
The integration of data points belonging to different experiments with their own data densities requires a careful treatment to avoid biases and to properly account for correlated systematic uncertainties within the same experiment and between different experiments, as well as within and between different channels. Quadratic interpolation (splines) of adjacent data points is performed for each experiment, and a local combination in form of a weighted average of the interpolations is computed in bins of 1$\;$MeV, or in narrower bins for the $\omega$ and $\phi$ resonances. 

The  uncertainties on the combined dataset, the data integration and the phenomenological fit are computed using large numbers of pseudo-experiments. 
These are generated taking into account all measurement uncertainties and their correlations. While this treatment guarantees a proper propagation of uncertainties, the resulting precision of the combination still depends on the chosen test statistic: a poor choice (e.g., an arithmetic instead of a weighted average) would lead to poor precision, while an aggressive choice (e.g., exploiting the available correlation information globally over the full spectrum, thereby benefiting from constraints among different energy regimes\footnote{Systematic uncertainties are based on estimates which are impacted by imponderables regarding size and correlation among measurements, in particular  uncertainties due to theoretical  modelling. Systematic uncertainties are often evaluated in relatively wide mass ranges, the event topology may evolve between measurements performed at different centre-of-mass energies (affecting for example the acceptance and tracking efficiency) as does the background composition, systematic uncertainties due to  trigger and tracking may be correlated, etc. It is therefore important to treat systematic uncertainties and their correlations with care and avoid the use of long-range correlations to constrain measurements among different centre-of-mass energies. Ambiguities in systematic uncertainties and their correlations have been studied in other experimental areas and  different ``configurations"/``scenarios" of uncertainties were proposed~\cite{Aad:2014bia, Aaboud:2017dvo, Aaboud:2017wsi}. }) could lead to an optimistic precision claim with the risk of undercoverage with respect to the (unknown) truth. 
To avoid either case, we employ a test statistic that only relies on local measurement uncertainties and correlations to combine datasets in a given bin.\footnote{This information on the uncertainties and correlations is used on slightly wider ranges, of typically up to a couple of 100~MeV, when {\it averaging regions} are defined in order to account for the difference between the point-spacing and bin-sizes for the various experiments~\cite{g209}. In this procedure the systematic uncertainties are not constrained, but rather directly propagated from each input measurement to the {\it averaging regions} and then to the fine bins.}
As stated above, the uncertainty in each combined bin and the correlation among bins are evaluated using pseudo-experiments generated with the full correlation information. Correlations between channels are accounted for by propagating individually the common systematic uncertainties.\footnote{A number of 15 such uncertainties are accounted for in the current study. Typical examples are the luminosity uncertainties, if the data stem from the same experimental facility but measure different channels, and uncertainties related to radiative corrections.} 



Where results from different datasets are locally inconsistent, the combined uncertainty is rescaled according to the local $\chi^2$ value and number of degrees of freedom following the PDG prescription~\cite{pdg}. Such inconsistencies are currently limiting the precision of the combination in the dominant $\pi^+\pi^-$ channel as well as in the $\Kp\Km$ channel  (see discussions below). In most exclusive channels the largest weight in the combination is provided by BABAR data. 

Closure tests with known distributions have been performed in the dominant \pipi channel to validate both the combination and integration procedures.
