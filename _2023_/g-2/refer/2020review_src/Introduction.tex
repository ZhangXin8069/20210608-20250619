\section{~Introduction}
\label{sec:Introduction}

The Standard Model (SM) predictions of the anomalous magnetic moment of the muon, 
$\amu=(g_\mu-2)/2$, with $g_\mu$ the muon gyromagnetic factor, 
and of the running electromagnetic coupling constant, $\alpha(s)$,  an important  ingredient 
of electroweak theory, are 
limited in precision by    hadronic vacuum polarisation (HVP) contributions. 
The dominant hadronic terms can be calculated with the use of experimental 
cross-section data, involving \ee annihilation to hadrons, and perturbative QCD to evaluate  energy-squared dispersion integrals ranging from the $\piz\gamma$ 
threshold to infinity. The  kernels occurring in these integrals 
emphasise low photon virtualities, owing to the $1/s$ descent of the cross section, 
and, in case of \amu, to an additional $1/s$ suppression. About 73\% 
of the lowest order hadronic contribution to \amu and 58\% of the total uncertainty-squared
are given by the $\ppg$ final state,\footnote
{Throughout this paper, final state photon radiation is implied for all 
   hadronic final states.
} 
while this channel amounts to only 12\% of the hadronic contribution to $\alpha(s)$ 
at $s=\mZ^2$~\cite{dhmz2017}.

In this work, we reevaluate the lowest-order hadronic contribution, \amuhadLO, to 
the muon magnetic anomaly, and the hadronic contribution, \dahadZ, to the running 
\aZ at the $Z$-boson mass using newest $\ee \to {\rm hadrons}$ cross-section 
data and updated techniques. In particular, we perform a phenomenological fit to supplement less precise data in the low-energy domain up to 0.6$\;$GeV. We also reconsider the systematic uncertainty in the \pipi channel in view of discrepancies among the most precise datasets.


All the experimental contributions are evaluated using the software package
HVPTools~\cite{g209}. To these are added  narrow resonance contributions evaluated 
analytically, and continuum contributions computed using perturbative QCD. 
